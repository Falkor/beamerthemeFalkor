% =============================================================================
% File:  sample_slides.tex --  Example of the use of the Falkor Beamer theme
% Author(s): Sebastien Varrette <Sebastien.Varrette@uni.lu>
% Time-stamp: <Thu 2012-09-27 20:58 svarrette>
% 
% Copyright (c) 2012 Sebastien Varrette <Sebastien.Varrette@uni.lu>
% .             http://varrette.gforge.uni.lu
% 
% For more information:
% - LaTeX: http://www.latex-project.org/
% - Beamer: https://bitbucket.org/rivanvx/beamer/
% =============================================================================
\documentclass{beamer}
% \documentclass[draft]{beamer}

%%%%%%%%%%%%%%%%%%%%%%%%%%%% Style configuration %%%%%%%%%%%%%%%%%%%%%%%%%%%%%%
\usepackage[english]{babel}
\usepackage{ae,url}
\usepackage{epsfig}
\usepackage{textcomp}
% \usepackage{graphicx}
\usepackage{amsmath, amsthm}
\usepackage{amsfonts,amssymb}
\usepackage{float}
\usepackage{floatflt}         % floating figures
\usepackage{multirow}
\usepackage{rotating}
\usepackage{xspace}
\usepackage{listings}
\usepackage{acronym}
\usepackage[official]{eurosym} % Euro symbol

\usepackage{colortbl}
\graphicspath{{images/}} % Add this directory to the searched paths for graphics

\bibliographystyle{acm}

% Listing stuff
\newenvironment{cmdline}
{\vspace{-0.5em}\begin{flushleft}\begin{scriptsize}}
        {\end{scriptsize}\end{flushleft}\vspace{-0.5em}}

\newcommand{\promptline}[2]{~~~~\textcolor{progressbar@fgblue}{\texttt{{\tiny #1\textnormal{\$>}}}} \texttt{#2}}
\newcommand{\cmdlineentry}[1]{\promptline{}{#1}}
\newcommand{\cmdlinefrontend}[1]{\promptline{(frontend)}{#1}}
\newcommand{\cmdlinenode}[1]{\promptline{(node)}{#1}}
\newcommand{\cmdlinecomment}[1]{\hfill{\tiny \textcolor{red}{\textit{\# #1}}}}


%%%%%%%%%%%%%%%%%%%%%%%%%%%%%%%%%%%%%%%% 
% Beamer Specific options
%%%%%%%%%%%%%% 
\definecolor{lightgray}{gray}{0.95}
\definecolor{trustcolor}{rgb}{0.1,0.3,0.7}
\definecolor{secplan}{rgb}{0.2,0.2,0.7}
\definecolor{beamerblue}{rgb}{0.2,0.2,0.7}

\hypersetup{
  colorlinks,%
  citecolor=beamerblue,%
  linkcolor=beamerblue,%
  urlcolor=beamerblue
}



% The key part to use my theme
\usetheme{Falkor}

% format the beginning of each new section
\AtBeginSection[]
{
  \frame{
    \frametitle{Summary}
    {\scriptsize\tableofcontents[currentsection]}
  }
  \addtocounter{framenumber}{-1}
}

\renewcommand{\epsilon}{\varepsilon}

%%%%%%%%%%%%%%%%%%%%%%%%%%%% Header %%%%%%%%%%%%%%%%%%%%%%%%%%%%%%
\title{The \texttt{Falkor} \LaTeX Beamer Style}
\subtitle{Overview and Usage}

\author{S\'ebastien Varrette}
\institute[CSC research unit]{
  Computer Science and Communications (CSC) Research Unit,
  University of Luxembourg, Luxembourg
}

\pgfdeclareimage[height=0.8cm]{logo}{Images/logo_UL.pdf}
\logo{\pgfuseimage{logo}}
\date{}

%%%%%%%%%%%%%%%%%%%%%%%%%%%%%% Body %%%%%%%%%%%%%%%%%%%%%%%%%%%%%%%
\begin{document}

\begin{frame}
    \vspace{2.5em}
    \titlepage
\end{frame}


\section{Usage}
\frame{
  \frametitle{Sample usage}

  \begin{block}{Bootstrap a new working directory}
      \begin{itemize}
        \item Get the latest version on
          \href{https://github.com/Falkor/beamerthemeFalkor}{Github}
          \begin{cmdline}
              \cmdlineentry{cd /path/to/cloning/dir}\\
              \cmdlineentry{git clone https://github.com/Falkor/beamerthemeFalkor.git}
          \end{cmdline}
        \item Duplicate the sample structure of the freshly cloned repository
          \begin{cmdline}
              \cmdlineentry{cd /path/to/working/dir}\\
              \cmdlineentry{rsync -avzu --exclude "*.git" $\backslash$\\ ~~~~~/path/to/cloning/dir/beamerthemeFalkor/ .}
          \end{cmdline}
        \item Copy and adapt the sample document \texttt{sample\_slides.tex}
          \begin{itemize}
            \item[$\hookrightarrow$] run \texttt{make} to generate \texttt{sample\_slides.pdf}
          \end{itemize}
      \end{itemize}
  \end{block}
}

\frame{
  \frametitle{Sample usage}

  \begin{block}{Git sub-modules approach}
      \begin{itemize}
        \item Assuming you want to use the theme in an existing git repo
          \begin{cmdline}
              \cmdlineentry{cd /path/to/working/dir}\\
              \cmdlineentry{git submodule add $\backslash$\\
                ~~~~~https://github.com/Falkor/beamerthemeFalkor.git
                $\backslash$\\
                ~~~~~.beamerthemeFalkor}\\
              \cmdlineentry{ln -s
                .beamerthemeFalkor/beamerthemeFalkor.sty .}
          \end{cmdline}
      \end{itemize}
  \end{block}
}

\frame{
  \frametitle{Changing the logo}

  \begin{itemize}
    \item The logo used by the theme is \texttt{images/slide\_image.jpg}
    \item To use another logo:
      \begin{cmdline}
          \cmdlineentry{cd images}\\
          \cmdlineentry{wget http://path/to/myimage.jpg}\\
          \cmdlineentry{ln -sf myimage.jpg slide\_image.jpg}\\
      \end{cmdline}
  \end{itemize}
}

\section{Conclusion}

% ............
\begin{frame}
    \frametitle{Conclusion}

    \begin{itemize}
      \item Summary point 1
      \item Summary point 2
    \end{itemize}

    \begin{block}{Perspectives}
        \begin{itemize}
          \item Improve point 1
          \item Improve point 2
        \end{itemize}
    \end{block}


\end{frame}


% ======================== END =========================

\section*{Thank you for your attention...}
\frame{
  \frametitle{Questions?}
  \begin{center}
      \includegraphics[scale=0.2]{question.jpg}
  \end{center}

  {\tiny
    \tableofcontents

  }
}

\newcounter{finalframe}
\setcounter{finalframe}{\value{framenumber}}

% \appendix

\frame{
  \frametitle{Appendix}

  \begin{acronym}\setlength\itemsep{-0.3em}
      \acro{DFT}{Discrete Fourier Transform}
      \acro{EA}{Evolutionary Algorithm}
      \acro{PRNG}{[Pseudo]-Random Number Generator}
      \acro{UL}{University of Luxembourg}
  \end{acronym}

  \textit{*Note: notice the slide number below...}
}

%.......
\frame{
  \frametitle{Another appendix slide}

  \textit{Note again the slide number below...}
  
}


\setcounter{framenumber}{\value{finalframe}}

\end{document}

% ~~~~~~~~~~~~~~~~~~~~~~~~~~~~~~~~~~~~~~~~~~~~~~~~~~~~~~~~~~~~~~~~
% eof
% 
% Local Variables:
% mode: latex
% mode: flyspell
% mode: auto-fill
% fill-column: 80
% End:
