% =============================================================================
% File:  markdown.tex --  Example of the use of the Falkor Beamer theme together with Markdown content
% Time-stamp: <Fri 2024-04-26 17:23 svarrette>
%
% Copyright (c) 2012-2024 Sebastien Varrette <Sebastien.Varrette@gmail.com>
%
% For more information:
% - LaTeX: http://www.latex-project.org/
% - Beamer: https://bitbucket.org/rivanvx/beamer/
% - LaTeX symbol list:
% http://www.ctan.org/tex-archive/info/symbols/comprehensive/symbols-a4.pdf
%
% Latest version of these files can be found on Github:
% https://github.com/Falkor/beamerthemeFalkor
% =============================================================================

\documentclass[aspectratio=169]{beamer}
% \documentclass[draft]{beamer}
\usepackage{_style}

% The key part to use my theme -- if you precise nothing, the image that
% illustrate the slides is assumed to be images/slides_image.jpg
\usetheme[image=images/logo_github.png]{Falkor}

% Not integrated in my theme as not everybody wants that
\AtBeginSection[]
{
  \frame{
    \frametitle{Summary}
    {\scriptsize\tableofcontents[currentsection]}
  }
}

\graphicspath{{images/}} % Add this directory to the searched paths for graphics


%%%%%%%%%% Header %%%%%%%%%%%%
\title{The \texttt{Falkor} \LaTeX Beamer Style}
\subtitle{An example using Markdown content and pandoc}

\author{Firstname Lastname}
\institute[Short institute]{
  Long institute
}
% Mandatory to **declare** a logo to be placed on the bottom right -- normally the
% university logo. ADAPT ACCORDINGLY:
\pgfdeclareimage[height=0.8cm]{logo}{images/logo_UL.pdf}

\date{}

%%%%%%%%%%%%% Body %%%%%%%%%%%%%%%
\begin{document}

\begin{frame}
  \vspace{2.5em}
  \titlepage
\end{frame}

% .......
\frame{
  \begin{center}
    \textbf{Latest versions available on
      \href{https://github.com/Falkor/}{Github}}:
    \vfill
    \begin{description}
      \item[Beamer theme Falkor:] \hfill
      \myurl{https://github.com/Falkor/beamerthemeFalkor}
      \item[Generic Makefiles:] \hfill
      \myurl{https://github.com/Falkor/Makefiles}
    \end{description}
  \end{center}
}

% ......
\frame{
  \frametitle{Summary}
  {\scriptsize
    \tableofcontents
  }
}

%===========================
\section{Instructions}

%.......
\frame[containsverbatim]{
  \frametitle{Markdown-based Beamer Workflow}

  \begin{itemize}
    \item Install \href{http://pandoc.org/}{pandoc}
    \item Rely on my \LaTeX\ \texttt{\href{https://github.com/Falkor/Makefiles/blob/devel/latex/Makefile}{Makefile}} (or define your own)
    \item Rely on a main file in \LaTeX
    \begin{itemize}
      \item split markdown content in individual Markdown files
      \item these file \textbf{\alert{SHOULD HAVE}} the \texttt{.md} extension
      \item my \texttt{\href{https://github.com/Falkor/Makefiles/blob/devel/latex/Makefile}{Makefile}} compile all \texttt{.md} files into \texttt{.md.tex} \LaTeX\ files using:\\[1em]
      \begin{cmdline}
        \cmdlineentry{pandoc --from markdown --to beamer --slide-level 3 \textbackslash \\ ~~~~~~~~~~~ -o <filename>.md.tex <filename>.md}\\
      \end{cmdline}
    \end{itemize}
    \item Simply include the markdown files using
    \texttt{\textbackslash input\{<filename>.md\}}
    \begin{itemize}
      \item in practice, \texttt{<filename>.md.tex} is considered for inclusion
    \end{itemize}
    \item \emph{Ex:} the next section was generated by:
    \begin{lstlisting}[language=tex,columns=fixed,basicstyle=\tiny,emph={\input}]
      \input{_content_in_markdown.md} % the .tex extension is
                                      % automatically added
    \end{lstlisting}
    \begin{itemize}
      \item Markdown sources are provided in appendix
    \end{itemize}
  \end{itemize}
}

\input{_content_in_markdown.md}

% ---------------------------------------
\section{Markdown Sources of these slides}

\frame[containsverbatim,allowframebreaks]{
  \frametitle{Markdown Sources}
  \lstinputlisting[basicstyle=\tiny]{_content_in_markdown.md.txt}
}


% ======================== END =========================
\section*{Thank you for your attention...}
\frame{
  \frametitle{Questions?}
  % ~~~~~~~~~~~~~~
  \begin{columns}
    \column{0.5\textwidth}
    % \emph{Contact}\\
           {\tiny
             \emph{Firstname Lastname}\\
             ~~~~ \textit{mail:} \href{mailto:firstame.lastname@domain.com}{firstname.lastname@domain.com}\\
             ~~~~ Street / Local address\\
             ~~~~ Postal code, country

    }
    \column{0.5\textwidth}
    % \scalebox{8}{\emph{?}}
    \includegraphics[width=1.5in]{question.jpg}
  \end{columns}
  % Below is the table of content over 2 columns
  \vfill
  \begin{multicols}{2}
    {\tiny \tableofcontents}
  \end{multicols}

}

\newcounter{finalframe}
\setcounter{finalframe}{\value{framenumber}}

% %.......
% \frame{
%   \frametitle{}
%   \vfill
%   \centering \LARGE Appendix\footnote{notice the slide number below...}
%   \vfill
% }


\setcounter{framenumber}{\value{finalframe}}

\end{document}

% ~~~~~~~~~~~~~~~~~~~~~~~~~~~~~~~~~~~~~~~~~~~~~~~~~~~~~~~~~~~~~~~~
% eof
%
% Local Variables:
% mode: latex
% mode: flyspell
% mode: visual-line
% TeX-master: "."
% End:
