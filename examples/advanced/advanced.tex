% =============================================================================
% File:  advanced.tex --  Example of the use of the Falkor Beamer theme
% Author(s): Sebastien Varrette <Sebastien.Varrette@uni.lu>
% Time-stamp: <Mon 2015-06-15 12:07 svarrette>
% 
% Copyright (c) 2012-2015 Sebastien Varrette <Sebastien.Varrette@uni.lu>
% 
% For more information:
% - LaTeX: http://www.latex-project.org/
% - Beamer: https://bitbucket.org/rivanvx/beamer/
% - LaTeX symbol list:
% http://www.ctan.org/tex-archive/info/symbols/comprehensive/symbols-a4.pdf
% 
% Latest version of these files can be found on Github:
% https://github.com/Falkor/beamerthemeFalkor
% =============================================================================

\documentclass{beamer}
% \documentclass[draft]{beamer}
\usepackage{_style}

% The key part to use my theme -- if you precise nothing, the image that
% illustrate the slides is assumed to be images/slides_image.jpg
\usetheme[image=images/logo_github.png]{Falkor}

% Not integrated in my theme as not everybody wants that
\AtBeginSection[]
{
  \frame{
    \frametitle{Summary}
    {\scriptsize\tableofcontents[currentsection]}
  }
}

\graphicspath{{images/}} % Add this directory to the searched paths for graphics


%%%%%%%%%% Header %%%%%%%%%%%%
\title{The \texttt{Falkor} \LaTeX Beamer Style}
\subtitle{A Complete Example with Instructions}

\author{S\'ebastien Varrette}
\institute[PCOG Research unit]{
  Parallel Computing and Optimization Group (\href{http://pcog.uni.lu}{PCOG}),
  University of Luxembourg (\href{http://www.uni.lu}{UL}), Luxembourg
}

% Mandatory to **declare** a logo to be placed on the bottom right -- normally the
% university logo. ADAPT ACCORDINGLY:
\pgfdeclareimage[height=0.8cm]{logo}{images/logo_UL.pdf}

\date{}

%%%%%%%%%%%%% Body %%%%%%%%%%%%%%%
\begin{document}

\begin{frame}
  \vspace{2.5em}
  \titlepage
\end{frame}

% .......
\frame{
  \begin{center}
    \textbf{Latest versions available on
      \href{https://github.com/Falkor/}{Github}}:
    \vfill
    \begin{description}
      \item[Beamer theme Falkor:] \hfill
      \myurl{https://github.com/Falkor/beamerthemeFalkor}
      \item[Generic Makefiles:] \hfill
      \myurl{https://github.com/Falkor/Makefiles}
    \end{description}
  \end{center}
}

% ......
\frame{
  \frametitle{Summary}
  {\scriptsize
    \tableofcontents
  }
}

% ====================
\section{Instructions}

% ----------------------------------
\subsection{Brutal Copy approach}

% ~~~~~~
\frame{
  \frametitle{Brutal copy usage}

  \begin{block}{Bootstrap a new working directory}
    \begin{itemize}
      \item Get the latest version on
      \href{https://github.com/Falkor/beamerthemeFalkor}{Github}
      \begin{cmdline}
        \cmdlineentry{cd /path/to/cloning/dir}\\
        \cmdlineentry{git clone https://github.com/Falkor/beamerthemeFalkor.git}
      \end{cmdline}
      \item Duplicate the sample structure of the freshly cloned repository
      \begin{cmdline}
        \cmdlineentry{cd /path/to/working/dir}\\
        \cmdlineentry{rsync -avzu --exclude "*.git" $\backslash$\\ ~~~~~/path/to/cloning/dir/beamerthemeFalkor/example/advanced .}
      \end{cmdline}
      \item Copy and adapt the sample document \texttt{advanced.tex}
      \begin{itemize}
        \item you might wish to rename it
        \item run \texttt{make} to generate \texttt{<name>.pdf}
      \end{itemize}
    \end{itemize}
  \end{block}
}

% ------------------------------------------
\subsection{The Git Submodule approach}

% .......
\frame{
  \frametitle{[Better] Git Submodule approach}
  Assuming you have your sources under a Git repository:

  \begin{block}{Git sub-modules approach}
    \begin{itemize}
      \item Initiate the submodule
      \begin{itemize}
        \item assuming \texttt{ /path/to/working/dir} is your root Git directory
      \end{itemize}
      \begin{cmdline}
        \cmdlineentry{cd /path/to/working/dir}\\
        \cmdlineentry{git submodule add $\backslash$\\
          ~~~~~https://github.com/Falkor/beamerthemeFalkor.git
          $\backslash$\\
          ~~~~~.submodules/beamerthemeFalkor}\\
      \end{cmdline}
      \item bootstrap the sub-directory hosting your slides sources:
      \begin{cmdline}
        \cmdlineentry{cd path/to/slides}\\
        \cmdlineentry{ln -s ../../.. .root   \textit{\# ADAPT -- relative path to git root}}\\
        \cmdlineentry{ln -s $\backslash$ .root/.submodules/.beamerthemeFalkor/beamerthemeFalkor.sty .}
      \end{cmdline}
    \end{itemize}
  \end{block}

}

% ------------------------------------------------
\subsection{The \texttt{FalkorLib} approach}

% .......
\frame{
  \frametitle{[Even Better] Fully automatic}

  Assuming again you have your sources under a Git repository:

  \begin{block}{}
    \begin{itemize}
      \item Install the \href{https://github.com/Falkor/falkorlib}{FalkorLib}
      Ruby Gem (\textbf{\alert{AT LEAST $\geq$ v.0.6 }})
      \begin{itemize}
        \item See
        \href{https://github.com/Falkor/falkorlib\#installation}{instructions}
        \item Then you shall have access to the \texttt{falkor} executable
      \end{itemize}
      \item Bootstrap your Beamer slides using
      \begin{cmdline}
        \cmdlineentry{falkor new slides}\\
      \end{cmdline}
    \end{itemize}
  \end{block}
}

% ==============================
\section{Customizations}


\frame[containsverbatim]{
  \frametitle{Changing the image or the logo}

  \begin{block}{}
    \begin{itemize}
      \item You can easily set the image illustrating the slides
      \begin{itemize}
        \item use the \texttt{image=<path>} option of the \texttt{Falkor} Beamer theme
        \item adapt the path accordingly. \emph{Ex:}
      \end{itemize}
      \begin{lstlisting}[language=tex,basicstyle=\footnotesize,emph={usetheme}]
        \usetheme[image=images/logo_github.png]{Falkor}
      \end{lstlisting}
    \end{itemize}
  \end{block}

  \begin{block}{}
    \begin{itemize}
      \item Set the logo  using the
      \texttt{\textbackslash pgfdeclareimage\{logo\}} instruction
      \begin{itemize}
        \item the logo will appear in the bottom right
        \item you can thus set \texttt{height} and \texttt{width} the way you
        want
        \item the name of the declared PGF image \textbf{\alert{SHOULD BE}}
        \texttt{logo}. \emph{Ex:}
      \end{itemize}
      \begin{lstlisting}[language=TeX,basicstyle=\footnotesize,emph={pgfdeclareimage}]
        \pgfdeclareimage[height=0.8cm]{logo}{logo_UL.pdf}
      \end{lstlisting}
    \end{itemize}
  \end{block}
}

%===============================
\section{Using Markdown}

%.......
\frame{
  \frametitle{Using Markdown}
  
  \begin{itemize}
    \item \href{http://daringfireball.net/projects/markdown/}{Markdown}: easy-to-read, easy-to-write plain text format
    \begin{itemize}
      \item facilitates the generation of Beamer slides
      \item prevent the use of \LaTeX for simple slides
      \begin{itemize}
        \item using just \texttt{itemize} etc.. 
      \end{itemize}
      \item conversion from  \href{http://daringfireball.net/projects/markdown/}{Markdown} to \LaTeX Beamer is done by \href{http://pandoc.org/}{pandoc}
    \end{itemize}
  \end{itemize}
  \begin{alertblock}{}
    $\Rightarrow$ See \href{https://github.com/Falkor/beamerthemeFalkor/tree/master/examples/markdown}{examples/markdown} for a full example
  \end{alertblock}

}




% ======================== END =========================
\section*{Thank you for your attention...}
\frame{
  \frametitle{Questions?}
  % ~~~~~~~~~~~~~~
  \begin{columns}
    \column{0.5\textwidth}
    % \emph{Contact}\\
    {\tiny
      \emph{Sebastien Varrette}\\
      ~~~~ \textit{mail:} \href{mailto:sebastien.varrette@uni.lu}{sebastien.varrette@uni.lu}\\
      ~~~~ Office E-007\\
      ~~~~ Campus Kirchberg\\
      ~~~~ 6, rue Coudenhove-Kalergi\\
      ~~~~ L-1359 Luxembourg

    }
    \column{0.5\textwidth}
    % \scalebox{8}{\emph{?}}
    \includegraphics[width=1.5in]{question.jpg}
  \end{columns}
  % Below is the table of content over 2 columns
  \vfill
  \begin{multicols}{2}
    {\tiny \tableofcontents}
  \end{multicols}

}

\newcounter{finalframe}
\setcounter{finalframe}{\value{framenumber}}

%.......
\frame{
  \frametitle{}
  \vfill
  \centering \LARGE Appendix\footnote{notice the slide number below...}
  \vfill
}


% ---------------------------------------
\section*{Sources of these slides}

\frame[containsverbatim,allowframebreaks]{
  \frametitle{Complete Sources (\href{https://github.com/Falkor/beamerthemeFalkor/blob/master/examples/advanced/advanced.tex}{advanced.tex})}
  \lstinputlisting[firstline=18,columns=flexible,language=TeX,basicstyle=\tiny,
                   emph={\documentclass,\usetheme,\AtBeginSection,\frame,\frametitle,\title,\subtitle,\author,\institute,\pgfdeclareimage,\date,\begin,\end,\lstinputlisting}
                   ]{advanced.tex}

}

\setcounter{framenumber}{\value{finalframe}}

\end{document}

% ~~~~~~~~~~~~~~~~~~~~~~~~~~~~~~~~~~~~~~~~~~~~~~~~~~~~~~~~~~~~~~~~
% eof
% 
% Local Variables:
% mode: latex
% mode: flyspell
% mode: visual-line
% TeX-master: "."
% End:
