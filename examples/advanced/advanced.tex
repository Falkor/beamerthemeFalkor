% =============================================================================
% File:  advanced.tex --  Example of the use of the Falkor Beamer theme
% Time-stamp: <Fri 2024-04-26 17:12 svarrette>
%
% Copyright (c) 2012-2024 Sebastien Varrette <Sebastien.Varrette@gmail.com>
%
% For more information:
% - LaTeX: http://www.latex-project.org/
% - Beamer: https://bitbucket.org/rivanvx/beamer/
% - LaTeX symbol list:
% http://www.ctan.org/tex-archive/info/symbols/comprehensive/symbols-a4.pdf
%
% Latest version of these files can be found on Github:
% https://github.com/Falkor/beamerthemeFalkor
% =============================================================================

\documentclass[aspectratio=169]{beamer}
% \documentclass[draft]{beamer}
\usepackage{_style}

% The key part to use my theme -- if you precise nothing, the image that
% illustrate the slides is assumed to be images/slides_image.jpg
\usetheme[image=images/logo_github.png]{Falkor}

% Not integrated in my theme as not everybody wants that
\AtBeginSection[]
{
  \frame{
    \frametitle{Summary}
    {\scriptsize\tableofcontents[currentsection]}
  }
}

\graphicspath{{images/}} % Add this directory to the searched paths for graphics


%%%%%%%%%% Header %%%%%%%%%%%%
\title{The \texttt{Falkor} \LaTeX Beamer Style}
\subtitle{A more advanced example with instructions}

\author{Firstname Lastname}
\institute[Short institute]{
  Long institute
}
% Mandatory to **declare** a logo to be placed on the bottom right -- normally the
% university/organization logo. ADAPT ACCORDINGLY:
\pgfdeclareimage[height=0.8cm]{logo}{images/logo_UL.pdf}

\date{}

%%%%%%%%%%%%% Body %%%%%%%%%%%%%%%
\begin{document}

\begin{frame}
  \vspace{2.5em}
  \titlepage
\end{frame}

% .......
\frame{
  \begin{center}
    \textbf{Latest versions available on
      \href{https://github.com/Falkor/}{Github}}:
    \vfill
    \begin{description}
      \item[Beamer theme Falkor:] \hfill
      \myurl{https://github.com/Falkor/beamerthemeFalkor}
      \item[Generic Makefiles:] \hfill
      \myurl{https://github.com/Falkor/Makefiles}
    \end{description}
  \end{center}
}

% ......
\frame{
  \frametitle{Summary}
  {\scriptsize
    \tableofcontents
  }
}


% ==============================
\section{More complete setup instructions}


\frame[containsverbatim]{
  \frametitle{More complete setup instructions}

  \begin{lstlisting}[language=bash,basicstyle=\ttfamily\footnotesize,columns=fullflexible]
    # Assuming you have initiated a git repo to host your LaTeX Beamer slides
    cd path/to/project/repo
    git init # if not yet done
    # Add beamertheme Falkor as a git submodule
    git submodule add https://github.com/Falkor/beamerthemeFalkor \
        .submodules/beamerthemeFalkor
    git commit -a -s -m "Add beamerthemeFalkor as Git submodule"
    # Eventually repeat for the Falkor's makefiles
    git submodule add -b devel https://github.com/Falkor/Makefiles \
        .submodules/Makefiles
    # create a src directory hosting the sources of your slides
    mkdir -p  slides/<year>/<topic>/src  # /!\ ADAPT path accordingly
    cd  slides/<year>/<topic>/src
  \end{lstlisting}
}

\frame[containsverbatim]{
  \frametitle{More complete setup instructions}

  \begin{itemize}
    \item Populate the slides directory
  \end{itemize}
  \begin{lstlisting}[language=bash,basicstyle=\ttfamily\footnotesize]
    # From  slides/<year>/<topic>/src directory
    # Create a symlink to the git root directory
    ln -s $(git rev-parse --show-cdup) .root
    # simlink beamerthemeFalkor.sty from there
    ln -s .root/.submodules/beamerthemeFalkor/beamerthemeFalkor.sty beamerthemeFalkor.sty
    # Copy (and transferring all the symlinks as regular files with '-L') from the
    # 'examples/advanced' directory as template:
    rsync -avzu -L --exclude '.root' --exclude 'beamerthemeFalkor.sty' \
          .root/.submodules/beamerthemeFalkor/example/advanced/ .
    mv advanced.tex slides-<topic>.text # /!\ ADAPT filename accordingly
    make  # hopefully this should compile...
  \end{lstlisting}

}



% ==============================
\section{Customizations}


\frame[containsverbatim]{
  \frametitle{Changing the image or the logo}

  \begin{block}{}
    \begin{itemize}
      \item You can easily set the image illustrating the slides
      \begin{itemize}
        \item use the \texttt{image=<path>} option of the \texttt{Falkor} Beamer theme
        \item adapt the path accordingly. \emph{Ex:}
      \end{itemize}
      \begin{lstlisting}[language=tex,basicstyle=\footnotesize,emph={usetheme}]
        \usetheme[image=images/logo_github.png]{Falkor}
      \end{lstlisting}
    \end{itemize}
  \end{block}

  \begin{block}{}
    \begin{itemize}
      \item Set the logo  using the
      \texttt{\textbackslash pgfdeclareimage\{logo\}} instruction
      \begin{itemize}
        \item the logo will appear in the bottom right
        \item you can thus set \texttt{height} and \texttt{width} the way you
        want
        \item the name of the declared PGF image \textbf{\alert{SHOULD BE}}
        \texttt{logo}. \emph{Ex:}
      \end{itemize}
      \begin{lstlisting}[language=TeX,basicstyle=\footnotesize,emph={pgfdeclareimage}]
        \pgfdeclareimage[height=0.8cm]{logo}{logo_UL.pdf}
      \end{lstlisting}
    \end{itemize}
  \end{block}
}

%===============================
\section{Using Markdown}

%.......
\frame{
  \frametitle{Using Markdown}

  \begin{itemize}
    \item \href{http://daringfireball.net/projects/markdown/}{Markdown}: easy-to-read, easy-to-write plain text format
    \begin{itemize}
      \item facilitates the generation of Beamer slides
      \item prevent the use of \LaTeX for simple slides
      \begin{itemize}
        \item using just \texttt{itemize} etc..
      \end{itemize}
      \item conversion from  \href{http://daringfireball.net/projects/markdown/}{Markdown} to \LaTeX Beamer is done by \href{http://pandoc.org/}{pandoc}
    \end{itemize}
  \end{itemize}
  \begin{alertblock}{}
    $\Rightarrow$ See \href{https://github.com/Falkor/beamerthemeFalkor/tree/master/examples/markdown}{examples/markdown} for a full example
  \end{alertblock}

}




% ======================== END =========================
\section*{Thank you for your attention...}
\frame{
  \frametitle{Questions?}
  % ~~~~~~~~~~~~~~
  \begin{columns}
    \column{0.5\textwidth}
    % \emph{Contact}\\
    {\tiny
         \emph{Firstname Lastname}\\
         ~~~~ \textit{mail:} \href{mailto:firstame.lastname@domain.com}{firstname.lastname@domain.com}\\
         ~~~~ Street / Local address\\
         ~~~~ Postal code, country

    }
    \column{0.5\textwidth}
    % \scalebox{8}{\emph{?}}
    \includegraphics[width=1.5in]{question.jpg}
  \end{columns}
  % Below is the table of content over 2 columns
  \vfill
  \begin{multicols}{2}
    {\tiny \tableofcontents}
  \end{multicols}

}

\newcounter{finalframe}
\setcounter{finalframe}{\value{framenumber}}

%.......
\frame{
  \frametitle{}
  \vfill
  \centering \LARGE Appendix\footnote{notice the slide number below...}
  \vfill
}


% ---------------------------------------
\section*{Sources of these slides}

\frame[containsverbatim,allowframebreaks]{
  \frametitle{Complete Sources (\href{https://github.com/Falkor/beamerthemeFalkor/blob/master/examples/advanced/advanced.tex}{advanced.tex})}
  \lstinputlisting[firstline=18,columns=flexible,language=TeX,basicstyle=\tiny,
                   emph={\documentclass,\usetheme,\AtBeginSection,\frame,\frametitle,\title,\subtitle,\author,\institute,\pgfdeclareimage,\date,\begin,\end,\lstinputlisting}
                   ]{advanced.tex}

}

\setcounter{framenumber}{\value{finalframe}}

\end{document}

% ~~~~~~~~~~~~~~~~~~~~~~~~~~~~~~~~~~~~~~~~~~~~~~~~~~~~~~~~~~~~~~~~
% eof
%
% Local Variables:
% mode: latex
% mode: flyspell
% mode: visual-line
% TeX-master: "."
% End:
