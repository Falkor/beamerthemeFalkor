% =============================================================================
% File:  minimal.tex --  Example of the use of the Falkor Beamer theme
% Author(s): Sebastien Varrette <Sebastien.Varrette@uni.lu>
% Time-stamp: <Mon 2015-06-15 11:30 svarrette>
% 
% Copyright (c) 2012-2015 Sebastien Varrette <Sebastien.Varrette@uni.lu>
% 
% For more information:
% - LaTeX: http://www.latex-project.org/
% - Beamer: https://bitbucket.org/rivanvx/beamer/
% - LaTeX symbol list:
% http://www.ctan.org/tex-archive/info/symbols/comprehensive/symbols-a4.pdf
% 
% Latest version of these files can be found on Github:
% https://github.com/Falkor/beamerthemeFalkor
% =============================================================================

\documentclass{beamer}
% \documentclass[draft]{beamer}

% The key part to use my theme -- if you precise nothing, the image that
% illustrate the slides is assumed to be images/slides_image.jpg
\usetheme[image=logo_github.png]{Falkor}

% Not integrated in my theme as not everybody wants that
\AtBeginSection[]
{
  \frame{
    \frametitle{Summary}
    {\scriptsize\tableofcontents[currentsection]}
  }
}


%%%%%%%%%%%%%%%%%%%%%%%%%%%% Header %%%%%%%%%%%%%%%%%%%%%%%%%%%%%%
\title{The \texttt{Falkor} \LaTeX Beamer Style}
\subtitle{A Minimal Example}

\author{S\'ebastien Varrette}
\institute[PCOG Research unit]{
  Parallel Computing and Optimization Group (\href{http://pcog.uni.lu}{PCOG}),
  University of Luxembourg (\href{http://www.uni.lu}{UL}), Luxembourg
}

% Mandatory to **declare** a logo to be placed on the bottom right -- normally the
% university logo. ADAPT ACCORDINGLY:
\pgfdeclareimage[height=0.8cm]{logo}{logo_UL.pdf}

\date{}

%%%%%%%%%%%%%%%%%%%%%%%%%%%%%% Body %%%%%%%%%%%%%%%%%%%%%%%%%%%%%%%
\begin{document}

\begin{frame}
  \vspace{2.5em}
  \titlepage
\end{frame}

% .......
\frame{
  \begin{center}
    \textbf{Latest versions available on
      \href{https://github.com/Falkor/}{Github}}:
    \vfill
    \begin{description}
      \item[Beamer theme Falkor:] \hfill
      \myurl{https://github.com/Falkor/beamerthemeFalkor}
      \item[Generic Makefiles:] \hfill
      \myurl{https://github.com/Falkor/Makefiles}
    \end{description}
  \end{center}
}

% ......
\frame{
  \frametitle{Summary}
  {\scriptsize
    \tableofcontents
  }
}

% ===============================================
\section{Introduction}
% ===============================================

% .......
\frame{
  \frametitle{Introduction}

  \begin{itemize}
    \item Topic of the talk
    \begin{itemize}
      \item a sub item
    \end{itemize}
  \end{itemize}

  \begin{block}{A block -- Objective}
    \begin{enumerate}
      \item first objective
      \item second objective
    \end{enumerate}
  \end{block}

}

% ===============================================
\section{Related Work}
% ===============================================

\subsection{A First Topic}

% .......
\frame{
  \frametitle{State-of-the-art: First Topic}
  \hfill\includegraphics[width=4em]{logo_github.png}

  \begin{itemize}
    \item<+-> first item
    \item<+-> second item
    \begin{itemize}
      \item sub item
    \end{itemize}
  \end{itemize}

  \uncover<3>{
    \begin{alertblock}{}
      \centering
      That's and \alert{alertblock} to display important results
    \end{alertblock}
  }
}


\subsection{A second Topic}

% .......
\frame{
  \frametitle{State-of-the-art: Second Topic}

  \begin{exampleblock}{Example}
    An example block
  \end{exampleblock}

}

% ===============================================
\section{Experimental Setup}
% ===============================================

% ===============================================
\section{Experimental Validation}
% ===============================================

\subsection{Test Case 1}
\subsection{Test Case 2}


% ===============================================
\section{Conclusion}
% ===============================================

% ............
\begin{frame}
  \frametitle{Conclusion}

  \begin{itemize}
    \item Summary point 1
    \item Summary point 2
  \end{itemize}

  \begin{block}{Perspectives}
    \begin{itemize}
      \item Improve point 1
      \item Improve point 2
    \end{itemize}
  \end{block}


\end{frame}





% ======================== END =========================
\section*{Thank you for your attention...}
\frame{
  \frametitle{Questions?}
   % ~~~~~~~~~~~~~~
   \begin{columns}
       \column{0.5\textwidth}
       %\emph{Contact}\\
       {\tiny
         \emph{Sebastien Varrette}\\
         ~~~~ \textit{mail:} \href{mailto:sebastien.varrette@uni.lu}{sebastien.varrette@uni.lu}\\
         ~~~~ Office E-007\\
         ~~~~ Campus Kirchberg\\
         ~~~~ 6, rue Coudenhove-Kalergi\\
         ~~~~ L-1359 Luxembourg

       }
       \column{0.5\textwidth}
       %\scalebox{8}{\emph{?}}
       \includegraphics[width=1.5in]{question.jpg}
   \end{columns}
   % Below is the table of content over 2 columns
   \vfill
   \begin{multicols}{2}
     {\tiny \tableofcontents}
   \end{multicols}
   
}

\newcounter{finalframe}
\setcounter{finalframe}{\value{framenumber}}

% \appendix

\frame{
  \frametitle{Appendix}

  A first appendix page\footnote{notice the slide number below...}

}

% .......
\frame{
  \frametitle{Another appendix slide}

  \textit{Note again the slide number below...}

}


\setcounter{framenumber}{\value{finalframe}}

\end{document}

% ~~~~~~~~~~~~~~~~~~~~~~~~~~~~~~~~~~~~~~~~~~~~~~~~~~~~~~~~~~~~~~~~
% eof
% 
% Local Variables:
% mode: latex
% mode: flyspell
% mode: visual-line
% TeX-master: "."
% End:
