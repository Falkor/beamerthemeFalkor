% =============================================================================
% File:  minimal.tex --  Example of the use of the Falkor Beamer theme
% Author(s): Sebastien Varrette <Sebastien.Varrette@uni.lu>
% Time-stamp: <Sat 2015-06-13 12:19 svarrette>
% 
% Copyright (c) 2012-2015 Sebastien Varrette <Sebastien.Varrette@uni.lu>
% 
% For more information:
% - LaTeX: http://www.latex-project.org/
% - Beamer: https://bitbucket.org/rivanvx/beamer/
% - LaTeX symbol list:
% http://www.ctan.org/tex-archive/info/symbols/comprehensive/symbols-a4.pdf
% 
% Latest version of these files can be found on Github:
%    https://github.com/Falkor/beamerthemeFalkor
% =============================================================================

\documentclass{beamer}
% \documentclass[draft]{beamer}
\usepackage{_style}


% The key part to use my theme -- if you precise nothing, the logo is assumed to
% be images/slides_image.jpg
\usetheme[logo=logo_github.png]{Falkor}

% Not integrated in my theme as not everybody wants that
\AtBeginSection[]
{
  \frame{
    \frametitle{Summary}
    {\scriptsize\tableofcontents[currentsection]}
  }
}


%%%%%%%%%%%%%%%%%%%%%%%%%%%% Header %%%%%%%%%%%%%%%%%%%%%%%%%%%%%%
\title{The \texttt{Falkor} \LaTeX Beamer Style}
\subtitle{A Minimal Example}

\author{S\'ebastien Varrette}
\institute[CSC research unit]{
  Computer Science and Communications (CSC) Research Unit,
  University of Luxembourg, Luxembourg
}

% Mandatory to declare a logo to be placed on the bottom right -- normally the
% university logo.
% manner.
\pgfdeclareimage[height=0.8cm]{logo}{logo_UL.pdf}
\date{}

% \usebackgroundtemplate{
%   \parbox[c][\paperheight][b]{\paperwidth}{\includegraphics[width=\paperwidth]{UL-bottom2.png}}
% }



%%%%%%%%%%%%%%%%%%%%%%%%%%%%%% Body %%%%%%%%%%%%%%%%%%%%%%%%%%%%%%%
\begin{document}

\begin{frame}
    \vspace{2.5em}
    \titlepage
\end{frame}

% .......
\frame{
  \begin{center}
      \textbf{Latest versions available on
        \href{https://github.com/Falkor/}{Github}}:
      \vfill
      \begin{description}
        \item[Beamer theme Falkor:] \hfill
          \myurl{https://github.com/Falkor/beamerthemeFalkor}
        \item[Generic Makefiles:] \hfill
          \myurl{https://github.com/Falkor/Makefiles}
        \item[Git bootstrapping script:] \hfill
          \myurl{https://github.com/Falkor/Makefiles/blob/devel/scripts/}
      \end{description}
  \end{center}
}

% ......
\frame{
  \frametitle{Summary}
  {\scriptsize
    \tableofcontents
  }
}

% ===============================================
\section{Installation}

% ........................
\frame[containsverbatim]{
  \frametitle{Basic usage}

  \begin{block}{}%Bootstrap a new working directory}
      \begin{itemize}
        \item Get the latest version on
          \href{https://github.com/Falkor/beamerthemeFalkor}{Github}
          \begin{cmdline}
              \cmdlineentry{cd /path/to/cloning/dir}\\
              \cmdlineentry{git clone https://github.com/Falkor/beamerthemeFalkor.git}
          \end{cmdline}

        \item Copy \href{https://github.com/Falkor/beamerthemeFalkor/blob/master/beamerthemeFalkor.sty}{beamerthemeFalkor.sty} at the root of your \LaTeX document


        \item Place the following code on your \LaTeX\ file:
          \begin{lstlisting}[basicstyle=\tiny,numbers=none]
              \usetheme{Falkor}
          \end{lstlisting}

        \item That's all (normally).
          \begin{itemize}
              \itemhook you might want to use my \href{https://github.com/Falkor/Makefiles/blob/devel/latex/Makefile}{Generic Makefile for \LaTeX}
          \end{itemize}

          % \item Duplicate the sample structure of the freshly cloned repository
          %   \begin{cmdline}
          %       \cmdlineentry{cd /path/to/working/dir}\\
          %       \cmdlineentry{rsync -avzu --exclude "*.git" $\backslash$\\ ~~~~~/path/to/cloning/dir/beamerthemeFalkor/ .}
          %   \end{cmdline}
          % \item Copy and adapt the sample document \texttt{sample\_slides.tex}
          %   \begin{itemize}
          %     \item[$\hookrightarrow$] run \texttt{make} to generate \texttt{sample\_slides.pdf}
          %   \end{itemize}
      \end{itemize}
  \end{block}
}

% % .......
% \frame{
%   \frametitle{Full sample example \hfill{\tiny (\textit{i.e.} these slides)}}

%   \begin{block}{}
%       \begin{itemize}
%         \item To copy a full working example
%           \begin{cmdline}
%               \cmdlineentry{cd /path/to/cloning/dir}\\
%               \cmdlineentry{git clone https://github.com/Falkor/beamerthemeFalkor.git}\\
%               \cmdlineentry{cd /path/to/working/dir}\\
%               \cmdlineentry{rsync -avzu -L --exclude "*.git" $\backslash$\\~~~~~~~/path/to/cloning/dir/beamerthemeFalkor/ .}\\
%               \cmdlineentry{make}
%           \end{cmdline}
%       \end{itemize}
%   \end{block}
%   \begin{itemize}
%     \item This will generate the file \texttt{sample\_slides.pdf}.
%       \begin{itemize}
%           \itemhook adapt accordingly...
%       \end{itemize}
%   \end{itemize}

% }

% \frame{
%   \frametitle{The scripted appraoch}

%   \begin{block}{Git sub-modules approach}
%       \begin{itemize}
%         \item Assuming you want to use the theme in an existing git repo
%           \begin{cmdline}
%               \cmdlineentry{cd /path/to/working/dir}\\
%               \cmdlineentry{git submodule add $\backslash$\\
%                 ~~~~~https://github.com/Falkor/beamerthemeFalkor.git
%                 $\backslash$\\
%                 ~~~~~.beamerthemeFalkor}\\
%               \cmdlineentry{ln -s
%                 .beamerthemeFalkor/beamerthemeFalkor.sty .}
%           \end{cmdline}
%       \end{itemize}
%   \end{block}
% }

% \frame{
%   \frametitle{Changing the logo}

%   \begin{itemize}
%     \item The logo used by the theme is \texttt{images/slide\_image.jpg}
%     \item To use another logo:
%       \begin{cmdline}
%           \cmdlineentry{cd images}\\
%           \cmdlineentry{wget http://path/to/myimage.jpg}\\
%           \cmdlineentry{ln -sf myimage.jpg slide\_image.jpg}\\
%       \end{cmdline}
%   \end{itemize}
% }

% % ===============================================
% \section{Some example slides}

% ............
\frame{

  \frametitle{Objectives of our work}

  \begin{itemize}
    \item Better than assumptions/\textit{a-priori}: concrete models and
      experiments
  \end{itemize}

  \begin{alertblock}{}
      \begin{itemize}
        \item Evaluate impact of the underlying hypervisor
          \begin{itemize}
              \itemhook at the heart of \textbf{any} cloud middleware so far
              % \itemhook analysis of the most widespread virtualization
              % frameworks
              \itemhook \emph{lightweight},
              \emph{high-level} model of a \emph{virtualized} machine.
          \end{itemize}
        \item Evaluate a real HPC platform (or anything as close as possible)
          \begin{itemize}
              \itemhook concrete deployment on top of the Grid5000 platform
              \itemhook select benchmarking tools to reflect an HPC usage
          \end{itemize}
      \end{itemize}
  \end{alertblock}


  %~\vfill
  {\tiny
    \mycite{SBAC-PAD13} S. Varrette,  M. Guzek, V. Plugaru, J. E. Sanchez, and P. Bouvry.
    "\textit{HPC Performance and Energy-Efficiency of Xen, KVM and VMware Hypervisors}". In Proc. of the 25th IEEE Symposium on Computer Architecture and High Performance (SBAC-PAD'13), Oct 2013.
    % Do not forget the below space

  }

}


% % .......
% \frame[t]{
%   \frametitle{The Grid'5000 Testbed \hfill \myurl{http://www.grid5000.fr}}

%   \begin{itemize}
%     \item Large scale nation wide infrastructure \hfill\includegraphics{logo_G5K.png}
%       \begin{itemize}
%           \itemhook for large scale parallel and distributed computing research.
%       \end{itemize}

%   \end{itemize}
%   % ~~~~~~~~~~~~~~
%   \begin{columns}
%       \column{0.4\textwidth}
%       \includegraphics[scale=0.9]{Renater5-g5k.jpg}
%       \column{0.6\textwidth}
%       \begin{itemize}
%         \item 10 sites in France
%         \item \emph{Abroad}: Luxembourg, Porto Allegre
%         \item Total: \textbf{7896} cores over \textbf{26} clusters
%         \item 1-10GbE / Myrinet / Infiniband interconnect
%         \item \emph{Kadeploy} functionnality
%       \end{itemize}
%   \end{columns}
% }


% \frame[containsverbatim]{
%   \frametitle{A slide with listings}%\frametitle{Problem}
%   \begin{block}{A JavaScript program}
%       \begin{lstlisting}[basicstyle=\tiny,numbers=none]
%           function fibo(n)
%           {
%             if( n <= 1 )
%             {
%               return n;
%             }
%             var res = fibo(n-1) + fibo(n-2);
%             return res;
%           }
%           n = parseFloat(arguments[1])
%           nn = fibo(n)
%           print(nn)
%       \end{lstlisting}

%   \end{block}

% }

\section{Using Markdown}

\input{_using_markdown.md}

% \section{Conclusion}

% % ............
% \begin{frame}
%     \frametitle{Conclusion}

%     \begin{itemize}
%       \item Summary point 1
%       \item Summary point 2
%     \end{itemize}

%     \begin{block}{Perspectives}
%         \begin{itemize}
%           \item Improve point 1
%           \item Improve point 2
%         \end{itemize}
%     \end{block}


% \end{frame}





% % ======================== END =========================

% \section*{Thank you for your attention...}
% \frame{
%   \frametitle{Questions?}
%   \begin{center}
%       \includegraphics[scale=0.2]{question.jpg}
%   \end{center}

%   {\tiny
%     \tableofcontents

%   }
% }

% \newcounter{finalframe}
% \setcounter{finalframe}{\value{framenumber}}

% % \appendix

% \frame{
%   \frametitle{Appendix}

%   \begin{acronym}\setlength\itemsep{-0.3em}
%       \acro{DFT}{Discrete Fourier Transform}
%       \acro{EA}{Evolutionary Algorithm}
%       \acro{PRNG}{[Pseudo]-Random Number Generator}
%       \acro{UL}{University of Luxembourg}
%   \end{acronym}

%   \textit{*Note: notice the slide number below...}
% }

% % .......
% \frame{
%   \frametitle{Another appendix slide}

%   \textit{Note again the slide number below...}

% }


% \setcounter{framenumber}{\value{finalframe}}

\end{document}

%~~~~~~~~~~~~~~~~~~~~~~~~~~~~~~~~~~~~~~~~~~~~~~~~~~~~~~~~~~~~~~~~
% eof
%
% Local Variables:
% mode: latex
% mode: flyspell
% mode: visual-line
% TeX-master: "."
% End:
